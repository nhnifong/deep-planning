\section{System \F} \label{sec:f}
\begin{figure}
\begin{singlespace}
\begin{minipage}{.46\textwidth}
	\begin{center}Church style\end{center}
\def\baselinestretch{0}
\small
\begin{align*}
\textbf{term syntax} \\
t,s ::= &~ x           & \text{variable}    \\
      | &~ \l(x:A) . t & \text{abstraction} \\
      | &~ t ~ s       & \text{application} \\
      | &~ \L X    . t & \text{type abstraction} \\
      | &~ t [A]       & \text{type application} \\
\\
\textbf{type syntax} \\
A,B ::= &~ X           & \text{variable type}   \\
      | &~ A -> B      & \text{arrow type} \\
      | &~ \forall X.B & \text{forall type}   \\
\end{align*}
\[ \textbf{kinding \& typing contexts} \]\vspace*{-1em}
\begin{align*}\quad
\Delta ::= &~ \cdot \\
	 | &~ \Delta, x:A & (X\notin \dom(\Delta)) \\
\Gamma ::= &~ \cdot \\
	 | &~ \Gamma, x:A & (x\notin \dom(\Gamma)) \\
\end{align*}
\[ \textbf{kinding rules} \quad \framebox{$ \Delta |- A $} \]\vspace*{-1em}
\begin{align*}
& \inference[\sc TVar]{X \in \Delta}{\Delta |- X} \\
& \inference[\sc TArr]{\Delta |- A & \Delta |- B}{\Delta |- A -> B} \\
& \inference[\sc TAll]{\Delta,X |- B}{\Delta |- \forall X.B} \\
\end{align*}
\[ \textbf{typing rules} \quad \framebox{$ \Delta;\Gamma |- t : A $ } \]
\vspace*{-1em}
\begin{align*}
& \inference[\sc Var]{x:A \in \Gamma}{\Delta;\Gamma |- x:A} \\
& \inference[\sc Abs]{\Delta |- A & \Delta;\Gamma,x:A |- t : B}
	             {\Delta;\Gamma |- \l(x:A).t : A -> B} \\
& \inference[\sc App]{\Delta;\Gamma |- t : A -> B & \Delta;\Gamma |- s : A}
		     {\Delta;\Gamma |- t~s : B} \\
& \inference[\sc TyAbs]{\Delta,X;\Gamma |- t : B}
		       {\Delta;\Gamma |- \L X.t : \forall X.B} ~
		       (X\notin\FV(\Gamma)) \\
& \inference[\sc TyApp]{\Delta;\Gamma |- t : \forall X.B & \Delta |- A}
		       {\Delta;\Gamma |- t[A] : B[A/X]}
\end{align*}
\end{minipage}
\begin{minipage}{.46\textwidth}
	\begin{center}Curry style\end{center}
\def\baselinestretch{0}
\small
\begin{align*}
\textbf{term syntax} \\
t,s ::= &~ x           \\
      | &~ \l x    . t \\
      | &~ t ~ s       \\
        &~\phantom{| \L X}  \\
        &~\phantom{| t [A]\vspace*{.1em}} \\
\\
\textbf{type syntax} \\
A,B ::= &~ X \\
      | &~ A -> B \\
      | &~ \forall X . B \\
\end{align*}
\[ \textbf{kinding \& typing contexts} \]\vspace*{-1em}
\begin{align*}\quad
\Delta ::= &~ \cdot \\
	 | &~ \Delta, x:A & (X\notin \dom(\Delta)) \\
\Gamma ::= &~ \cdot \\
	 | &~ \Gamma, x:A & (x\notin \dom(\Gamma)) \\
\end{align*}
\[ \textbf{kinding rules} \quad \framebox{$ \Delta |- A $}\]\vspace*{-1em}
\begin{align*}
& \inference[\sc TVar]{X \in \Delta}{\Delta |- X} \\
& \inference[\sc TArr]{\Delta |- A & \Delta |- B}{\Delta |- A -> B} \\
& \inference[\sc TAll]{\Delta,X |- B}{\Delta |- \forall X.B} \\
\end{align*}
\[ \textbf{typing rules} \quad \framebox{$ \Delta;\Gamma |- t : A $ } \]
\vspace*{-1em}
\begin{align*}
& \inference[\sc Var]{x:A \in \Gamma}{\Delta;\Gamma |- x:A} \\
& \inference[\sc Abs]{\Delta |- A & \Delta;\Gamma,x:A |- t : B}
		     {\Delta;\Gamma |- \l x   .t : A -> B} \\
& \inference[\sc App]{\Delta;\Gamma |- t : A -> B & \Delta;\Gamma |- s : A}
		     {\Delta;\Gamma |- t~s : B} \\
& \inference[\sc TyAbs]{\Delta,X;\Gamma |- t : B}
		       {\Delta;\Gamma |- t : \forall X.B} ~
		       (X\notin\FV(\Gamma)) \\
& \inference[\sc TyApp]{\Delta;\Gamma |- t : \forall X.B & \Delta |- A}
		       {\Delta;\Gamma |- t : B[A/X]}
\end{align*}
\end{minipage}
~\\
\caption{System \F\ in Church style and Curry style}
\label{fig:f}
\end{singlespace}
\end{figure}

\begin{figure}
\paragraph{Reduction rules for the Church-style System \F}
\begin{align*}
& \inference[\sc RedBeta]{}{(\l(x:A).t)~s --> t[s/x]}
&&\inference[\sc RedTy]{}{(\L X   .t)[A] --> t[A/X]} \\
& \inference[\sc RedAbs]{t --> t'}{\l x   .t --> \l x   .t'}
&&\inference[\sc RedTyAbs]{t --> t'}{\L X   .t --> \L X   .t'} \\
& \inference[\sc RedApp1]{t --> t'}{t~s --> t'~s}
&&\inference[\sc RedTyApp]{t --> t'}{t[A] --> t'[A]} \\
& \inference[\sc RedApp2]{s --> s'}{t~s --> t~s'}
\end{align*}
\paragraph{Reduction rules for the Curry-style System \F}
\begin{align*}
& \inference[\sc RedBeta]{}{(\l x   .t)~s --> t[s/x]} \\
& \inference[\sc RedAbs]{t --> t'}{\l x   .t --> \l x   .t'} \\
& \inference[\sc RedApp1]{t --> t'}{t~s --> t'~s} \\
& \inference[\sc RedApp2]{s --> s'}{t~s --> t~s'}
\end{align*}
\caption{Reduction rules for System \F}
\label{fig:redf}
\end{figure}

System \F\ extends the type syntax of STLC with type variables ($X$)
and forall types ($\forall X.B$), which enable us to express polymorphic types
(see Figure \ref{fig:f}). However, System \F\ does not have a dedicated syntax
for ground types, such as the void type $\iota$ in STLC. In System \F, we can
populate types from forall types such as $\forall X.X$. This type is, in fact,
an encoding of the void type. We shall see that large class of datatypes are
encodable in System \F (\S\ref{sec:f:data})

Unlike in STLC, not all types constructed by the type syntax of System \F\
make sense. Since we have type variables in System \F, we need to
make sure that types are well-kinded. That is, we should make sure
that all the type variables appearing in types are properly bound by
universal quantifiers ($\forall$). For instance, consider the two types
$\forall X.X$ and $\forall X.X'$. Under the empty kinding context,
$\forall X.X$ is well-kinded since $X$ is bounded by $\forall$, but
$\forall X.X'$ is ill-kinded since $X'$ is an unbound type variable.
The kinding rules determine whether a type is well-kinded.
In the kinding rules and typing rules, the kinding context ($\Delta$)
keeps track of the bound type variables. The complete syntax, kinding rules,
and typing rules of System \F are illustrated in Figure \ref{fig:f}.
The left column describes the Church-style System \F\ and the right
column describes the Curry-style System \F. The reduction rules are
shown separately in Figure \ref{fig:redf}.

As in STLC, the term syntax for abstractions differs between the two styles.
The Church-style System \F\ has type annotations in abstractions but
the Curry-style System \F\ does not. Furthermore, the Church-style System \F\
has additional syntax for type abstractions and type applications. The syntax
for type abstractions ($\L X.t$) makes it explicit that the type of the term
should be generalized to a forall type. The syntax for type applications
($t[A]$) makes it explicit that the type of the term should be instantiated to
a specific type from a forall type. On the contrary, the Curry-style System \F\
has neither type abstractions nor type type applications in the term syntax.
So, it is implicit in Curry style where types are generalized and instantiated.
The differences in typing rules and reduction rules between the two styles
follow from this difference in the term syntax.

The typing rules \rulename{Var}, \rulename{Abs}, and \rulename{App} are
pretty much the same as in STLC except that we carry around the kinding context
($\Delta$) along with the typing context ($\Gamma$). What is new in System \F\
are the typing rules for type abstractions (\rulename{TyAbs}) and
type applications (\rulename{TyApp}), which enable us to introduce
forall types and instantiate forall types to a specific type.
In Church style, the use of these two rules \rulename{TyAbs} and
\rulename{TyAbs} are guided by the term syntax of type abstractions
($\L X.t$) and type applications ($t[A]$). So, the typing rules of
the Church-style System \F\ are syntax directed. In Curry style,
on the contrary, there are no term syntax to guide the use of the rules
\rulename{TyAbs} and \rulename{TyApp}. So, the typing rules of
the Curry-style System \F\ are not syntax directed.

The reduction rules for the Church-style System \F\ includes all
the reduction rules for the Church-style STLC. In addition,  there
are three more reduction rules (\rulename{RedTy}, \rulename{RedTyAbs},
and \rulename{RedTyApp}) involving type abstractions and type applications.

The reduction rules for the Curry-style System \F\ are exactly the same as
the reduction rules for the Curry-style STLC (Figure \ref{fig:stlc}) since
the terms of the Curry-style System \F\ are identical to 
the terms of the Curry-style STLC.

\subsection{Encoding datatypes in System \F}
\label{sec:f:data}
System \F\ is  powerful enough to encode a fairly large class of datatypes
within its type system. Encodings of well-known datatypes are listed in
Table \ref{tbl:dataF}. In System \F, we can encode non-recursive datatypes
that are either simply-typed (\eg, void, unit, and booleans)
or parametrized (\eg, pairs and sums).
More interestingly, we can also encode recursive datatypes
that are either simply-typed (natural numbers) or parametrized (lists).
All of these datatypes are classified as \emph{regular datatypes}.\footnote{
	We discuss the concept of regular datatypes,
	in contrast to non-regular datatypes, in \S\ref{sec:fw:data}. }
All regular datatypes that are not mutually recursive are encodable
in System \F. Encodings of mutually recursive datatypes require
more expressive type systems such as System \Fw\ (\S\ref{sec:fw}).

\begin{table}
\begin{tabular}{p{15mm}|lp{92mm}}
	\hline
void
& encoding of type	& $\textit{Void} = \forall X.X$ \\
& constructor		& \\
& eliminator		& $\l x.x$
	\\\hline
unit
& encoding of type	& $\textit{Unit} = \forall X.X -> X$	\\
& constructor		& $\mathtt{Unit} = \l x.x$ \\
& eliminator		& $\l x.\l x'.x\;x'$
	\\\hline
booleans
& encoding of type	& $\textit{Bool} = \forall X.X -> X -> X$ \\
& constructors		& $\mathtt{True} = \l x_1.\l x_2. x_1$,\quad
			$\mathtt{False} = \l x_1.\l x_2. x_2$ \\
& eliminator		& $\l x.\l x_1. \l x_2. x\;x_1\,x_2$ \qquad
			(\textbf{if} $x$ \textbf{then} $x_1$ \textbf{else} $x_2$)
	\\\hline
pairs
& encoding of type	& $ A_1\times A_2 = \forall X. (A_1 -> A_2 -> X) -> X$ \\
& constructor		& $\mathtt{Pair} = \l x_1.\l x_2.\l x'.x'\,x_1\,x_2$ \\
& eliminator		& $\l x.\l x'.x\;x'$ \par
			(by passing appropriate values to $x'$, we get\par
			~~$\textit{fst} = \l x.x(\l x_1.\l x_2.x_1)$,
			$\textit{snd} = \l x.x(\l x_1.\l x_2.x_2)$ )
	\\\hline
sums
& encoding of type	& $A_1+A_2 = \forall X. (A_1 -> X) -> (A_2 -> X) -> X$ \\
& constructors		& $\mathtt{Inl} = \l x. \l x_1. \l x_2 . x_1\,x$,\quad
			$\mathtt{Inr} = \l x. \l x_2. \l x_2 . x_2\,x$ \\
& eliminator		& $\l x.\l x_1. \l x_2. x\;x_1\,x_2$ \par
			(\textbf{case} $x$ \textbf{of}
				\{$\mathtt{Inl}~x' -> x_1\;x'$;
				  $\mathtt{Inr}~x' -> x_2\;x'$\})
	\\\hline
natural
& encoding of type	& $\textit{Nat} = \forall X. (X -> X) -> X -> X$ \\
numbers
& constructors		& $\mathtt{Succ} = \l x. \l x_s. \l. x_z. x_s (x\;x_s\,x_z)$,\par
			$\mathtt{Zero} = \l x_s. \l x_z. x_z$ \\
& eliminator		& $\l x.\l x_z. \l x_s.x\;x_s\,x_z$ \quad
			(iteration on natural num.)
	\\\hline
lists
& encoding of type	& $\textit{List}\;A = \forall X. (A -> X -> X) -> X -> X$ \\
& constructors		& $\mathtt{Cons} = \l x_a.\l x.\l x_c.\l x_n. x_c\,x_a\,(x\;x_c\,x_n)$,\par
			$\mathtt{Nil}\;\, = \l x_c.\l x_n.\l x_n$ \\
& eliminator		& $\l x.\l x_c. \l x_n.x\;x_c\,x_n$ \quad
			(\textit{foldr} $x_z$ $x_c$ $x$ in Haskell)
	\\\hline
\end{tabular}
\caption{Church encodings of regular datatypes being well-typed in System \F}
\label{tbl:dataF}
\end{table}

\citet{Church41} devised an encoding for natural numbers
in the untyped lambda calculus, based on the idea that the natural number $n$
is represented by a higher-order function ($\l x_s.\l x_z.x_s^n~x_z$), which
applies the first argument ($x_s$) $n$ times to the second argument ($x_z$).
Such an encoding of natural numbers is called Church numerals, named after
Alonzo Church. More generally, term encodings of the objects of datatypes
based on similar a idea are called Church encodings. In System \F,
these Church encoded terms can be well-typed by encoding the datatype
as a polymorphic type of System \F, as illustrated in Table \label{tbl:dataF}.
Such encodings for datatypes are called impredicative encodings
since it rely on the impredicative polymorphism of System \F.

Encodings of types, constructors, and eliminators for
several well-known datatypes are listed in Table \ref{tbl:dataF}.
We use the Curry-style System \F\ since the constructors and the eliminators
are exactly the same as the Church encodings in the untyped lambda calculus.
If we were to use the Church-style System \F, we would need to adjust
the constructors and the eliminators by adding type abstractions and
type applications in appropriate places. For example,
the constructor for $\textit{Unit}$ would be
$\mathtt{Unit} = \L X.\l x:X.x$ and the eliminator would be
$\l(x:\textit{Unit}).\L X.x[X]\;x'$.

Constructors produce objects of a datatype. Nullary constructors
(\aka\ constants) are objects by themselves. For example,
$\mathtt{Unit}$ (or, $\l x.x$) is a unit object,
$\mathtt{True}$ (or, $\l x_1.\l x_2. x_1$) is a boolean object,
$\mathtt{Zero}$ (or, $\l x_s. \l x_z. x_z$) is a natural number, and
$\mathtt{Nil}$ (or, $\l x_c.\l x_n.\l x_n$) is a list.
That is,
\[
|- \mathtt{Unit}:\textit{Unit} \qquad
|- \mathtt{True}:\textit{Bool} \qquad
|- \mathtt{Zero}:\textit{Nat} \qquad
|- \mathtt{Nil}:\forall X_a.\textit{List}\;X_a
\]
where $\textit{Unit}$ is a shorthand notation (or, type synonym)
for $\forall X.X -> X$, $Bool$ is for $\forall X.X -> X -> X$, and so on,
as described in Figure \ref{tbl:dataF}.
%%% say it is a function
Other (non-nullary) constructors expect some arguments to produce objects.
For example, $\mathtt{Pair}$ expects two arbitrary arguments to produce a pair,
$\mathtt{Succ}$ expects a natural number argument to produce another
natural number, and $\mathtt{Cons}$ expects a new element and a list as
arguments to produce another list. That is,
\begin{align*}
& |- \mathtt{Pair} : \forall X_1. \forall X_2. X_1 -> X_2 -> X_1\times X_2
&& |- \mathtt{Succ} : \textit{Nat} -> \textit{Nat} \\ &
|- \mathtt{Cons} : \forall X_a. X_a -> \mathit{List}\;X_a -> \mathit{List}\;X_a
\end{align*}
where ${X_1 \times X_2}$, $\mathit{Nat}$, and $\mathit{List\,X_a}$
are shorthand notations for encodings of the datatypes,
as described in Figure \ref{tbl:dataF}.
%%% maybe type syn ???

We can deduce the number of constructors for a datatype and the types
of those constructors from the impredicative encoding of the datatype.
The general form for the encodings of the simply-typed datatypes is:
\[D = \forall X. A_1 -> \cdots -> A_n -> X
	\qquad\text{where}~~ A_i = A_{i1} -> \cdots -> A_{ik} -> X \]
From the encoding of type above, we can deduce the following facts:
\begin{itemize}
\item $n$ is the number of constructors,
\item $k$ is the arity of the $i$th constructor, and
\item the type of the $i$th constructor is $A_i[D/X]$.
\end{itemize}
Note, $D$ is a shorthand notation for the entire encoding of the type.
So, $A_i[D/X]$ expands to $A_i[(\forall X. A_1 -> \cdots -> A_n -> X)/ X]$.
Here, the type variable $X$ in $A_i$ is substituted by a polymorphic type
$D$, or $(\forall X. \cdots)$. Recall that $X$ in $A_i$ comes from
the variable universally quantified in $D$. In System \F, we are able to
substitute the universally quantified type variable $X$ with
the very polymorphic type $D$ quantifying $X$. For this ability,
we say ``System \F\ is \emph{impredicative}''. Impredicative encodings
of datatypes rely on this impredicative nature (or, impredicativity)
of System \F.

Similarly, the general form for the encodings of the parametrized datatypes is
$D\,X_1 \cdots X_k = \forall X. A_1 -> \cdots -> A_n -> X$. Then,
the number of constructors is $n$ and the type of the $i$th constructor
is $\forall X_1.\cdots\forall X_n.A_i[D/X]$.

Eliminators consume objects of a datatype for computation.
An eliminator for a datatype expects an object of the datatype
as its first argument followed by arguments of computations
to be performed for each of the constructors. For instance, the eliminator
for void ($\l x.x$) expects only one argument since void has
no constructor, the eliminator for unit ($\l x.\l x'.x\;x'$) expect
two arguments since unit has one constructor, and the eliminator for booleans
($\l x.\l x_1. \l x_2. x\;x_1\,x_2$) expect three arguments since there are
two boolean constructors.

Eliminators examine the shape of the object (\ie, by which constructor it is
constructed) to perform the computation, which corresponds to the shape
of the object. For instance, the eliminator for booleans amounts to the
well-known if-then-else expression.
For recursive types, computations are performed recursively due to
the definition of the constructors that expect recursive arguments.
For instance, note that $(x~x_s~x_z)$ appearing in the definition of
$\mathtt{Succ}$ coincides with the body of the eliminator for natural numbers.
Eliminators for recursive types are also known as iterators or folds.

The impredicative encoding of a datatype specifies what it needs to eliminate
an object of the datatype. Recall the general form for the encodings of
the simply-typed datatypes:
\[D = \forall X. A_1 -> \cdots -> A_n -> X
	\qquad\text{where}~~ A_i = A_{i1} -> \cdots -> A_{ik} -> X \]
We can understand this encoding as follows:
\begin{quote}
To compute the result of type $X$ from an object of type $D$,
we need $n$ small computations, whose types are $A_1,\dots,A_n$.
When the object is constructed by $i$th constructor, we use the $i$th small
computation, whose type is $A_i$, that is, $A_{i1} -> \cdots -> A_{ik} -> X$.
This small computation gathers all the $k$ arguments supplied to
the $i$th constructor for the object construction, in order to
compute the result from those argument.
\end{quote}

For constants, the eliminator simply returns the argument being passed
to handle the constant, as it is. For example, the unit eliminator
$(\l x .\l x'.x\;x')$ will return what is passed into $x'$. That is,
\[   (\l x .\l x'.x\;x')~\mathtt{Unit}~s
 --> (\l x'.\mathtt{Unit}\;x')~s
 --> \mathtt{Unit}~s
 --> s
\] since $\texttt{Unit}=\l x.x$.
Similarly, the boolean eliminator $(\l x.\l x_1.\l x_2.x~x_1\;x_2)$
simply returns $x_1$ when $x$ is $\mathtt{True}$
and returns $x_2$ when $x$ is $\mathtt{False}$,
due to the definition of $\mathtt{True} = \l x_1.\l x_2. x_1$
and $\mathtt{False} = \l x_1.\l x_2. x_2$.

For non-nullary constructors, the argument being passed to the eliminator
to handle the constructor must be a function that collects the argument used
for the object construction. The pair eliminator $(\l x.\l x'.x\;x')$ expects
the argument $x'$ be of type $X_1 -> X_2 -> X$ where $X$ is the result type
you want to compute. For example, you may pass an addition function 
($\textit{Nat} -> \textit{Nat} -> \textit{Nat}\,$) to $x'$ to compute
the sum of the first element and the second element of a pair of
natural numbers ($\textit{Nat}\times\textit{Nat}\,$). We can define
selector functions $\mathit{fst}$ and $\mathit{snd}$ for pairs by
providing an appropriate
argument for $x'$ as described in Table \ref{tbl:dataF}.

The key idea behind Church encodings is that objects are defined by
how they will be eliminated. That is, the Church encoded objects
are, in fact, eliminators. Readers familiar with lambda calculi may have
noticed that all the eliminators in Table \ref{tbl:dataF} are
$\eta$-expansions of the identity function. The formulation of eliminators
in Table \ref{tbl:dataF} is just to emphasize how many arguments
each eliminator expects.

\subsection{Subject reduction and strong normalization}\label{sec:f:srsn}
We discuss two important properties of System \F, which holds in both
Church style and Curry style -- \emph{subject reduction} (\aka\
\emph{type preservation}) and \emph{strong normalization}.

\subsubsection*{Subject reduction}
The subject reduction theorem for System \F\ can be stated as follows:
\begin{theorem}[subject reduction]
$\inference{\Delta;\Gamma |- t : A  & t --> t'}{\Delta;\Gamma |- t' : A}$
\end{theorem}
We can prove subject reduction for System \F\, in a similar fashion
to the proof of subject reduction for STLC,
by induction on the derivation of the reduction rules.

In Church style, proof for all other cases except for the rules
\rulename{RedBeta} and \rulename{RedTy} are simply done by applying
the induction hypothesis. Since the typing rules in Church style are
syntax directed, there is no ambiguity on which typing rule to be used
in the derivation for a certain judgment. For the \rulename{RedBeta} case,
we use the substitution lemma. For proving the \rulename{RedTy} case,
we use the type substitution lemma. The substitution lemma and
the type substitution lemma are stated below:
\begin{lemma}[substitution]
$ \inference{\Delta;\Gamma,x:A |- t : B  & \Delta;\Gamma |- s : A}
	{\Delta;\Gamma |- t[s/x] : B} $
\end{lemma}
\begin{lemma}[type substitution]
$ \inference{\Delta,X;\Gamma |- t : B  & \Delta |- A}
	{\Delta;\Gamma |- t[A/X] : B[A/X]} ~ (X\notin\FV(\Gamma))$
\end{lemma}

In Curry style, the most interesting case is the \rulename{RedBeta} rule,
which we the substitution lemma to prove. The other rules are basically
done by applying the induction hypothesis, but there is a little complication
in the proof, compared to the proof in Church style, since the typing rules
are not syntax directed. Although we have less rules to consider than
the Church-style STLC, we need to deal with the ambiguity of which rule
to apply for a typing judgement. The ambiguity is due to the rules
\rulename{TyAbs} and \rulename{TyApp}.

An alternative way to prove subject reduction for the Curry-style STLC is
by translating the subject reduction property of the Curry-style STLC into
the subject reduction property of the Church-style STLC. That is, we extract
a Church-style term from a typing derivation in Curry style. It is not
difficult to see that each typing derivation in Curry style corresponds to
a unique Church-style term, and, that a reduction step in Curry style
corresponds to one or more reduction steps in Church style.\footnote{
It is not always one step to one step since the reduction rules
{\sc RedTyAbs} and {\sc RedTyApp} in Church style correspond to
zero reduction step in Curry style.}

\subsubsection*{Strong Normalization}
\begin{figure}
\begin{singlespace}
\begin{description}
\item[Interpretation of types] as saturated sets of normalizing terms
	whose free type variables are substituted according to
	the given type valuation ($\xi$):
\begin{align*}
[| X |]_\xi           &= \xi(X) \\ 
[| A -> B |]_\xi      &= [|A|]_\xi -> [|B|]_\xi \\
[| \forall X.B |]_\xi &= \bigcap_{\mathcal{A}\in\SAT} [|B|]_{\xi[X\mapsto\mathcal{A}]} \qquad\qquad\qquad (X\notin\dom(\xi))
\end{align*}
\item[Interpretation of kinding and typing contexts]
       as sets of type valuations and term valuations ($\xi$ and $\rho$):
\begin{align*}
[| \Delta        |] &= \dom(\Delta) -> \SAT \\
[| \Delta;\Gamma |] &= \{ \xi;\rho \mid \xi\in[|\Delta|], \rho\in[|\Gamma|]_\xi \} \\
[| \Gamma        |]_\xi\ &= \{ \rho \in \dom(\Gamma) -> \SN \mid \rho(x)=[|\Gamma(x)|]_\xi ~\text{for all}~x\in\dom(\Gamma) \}
\end{align*}
\item[Interpretation of terms]
	as terms themselves whose free variables are substituted according to
	the given pair of type and term valuation ($\xi$;$\rho$):
\begin{align*}
[| x      |]_{\xi;\rho} &= \rho(x) \\
[| \l x.t |]_{\xi;\rho} &= \l x . [|t|]_{\xi;\rho} \qquad (x\notin\dom(\rho)) \\
[| t ~ s  |]_{\xi;\rho} &= [| t |]_{\xi;\rho} ~ [| s |]_{\xi;\rho}
\end{align*}
\end{description}
\caption[Interpretation of System \F\ for proving strong normalization]
	{Interpretation of types, kinding and typing contexts, and terms
		of System \F\ for the proof of strong normalization}
\label{fig:interpF}
\end{singlespace}
%% \vspace*{.3em}\hrule
\end{figure}
For the proof of strong normalization, we use the same proof strategy of
interpreting types as saturated sets of normalizing terms as in the proof of
strong normalization of STLC in \S\ref{sec:stlc:srsn}. The interpretation of
types, contexts, and terms of System \F\ are illustrated in
Figure \ref{fig:interpF}. Since we have type variables we need a type valuation
($\xi$), which maps type variables to interpretations of types. So, the
interpretation of types are indexed by the type valuation ($\xi$), and
the interpretation of terms are indexed by the pair of term and type valuations
($\xi;\rho$). A type valuation $\xi$ is a function from $\dom(\Delta)$,
the set of bound type variables in $\Delta$, to $\SAT$, the set of all
saturated sets.

Any type interpretation is a saturated set. Since $\xi$ maps a type variable
to a saturated set, $[|X|]_\xi \in \SAT$. We know $[|A -> B|]_\xi \in \SAT$
since saturated sets are closed under the arrow operation ($->$), as we
mentioned in \S\ref{sec:stlc:srsn}. $[|\forall X.B|]_\xi \in \SAT$ since
it is known that saturated sets are closed under set indexed intersection.

The proof of strong normalization amounts to proving the following theorem:
\begin{theorem}[soundness of typing]
$ \inference{\Delta;\Gamma|- t:A & \xi;\rho \in [|\Delta;\Gamma|]}
	    {[|t|]_{\xi;\rho} \in [|A|]_\xi} $
\end{theorem}
\begin{proof}
We prove by induction on the typing derivation ($\Delta;\Gamma|- t:A$).
\paragraph{Case (\rulename{Var})}
It is trivial to show that
$ \inference{\Delta;\Gamma |- x:A & \xi;\rho \in [|\Delta;\Gamma|]}
	{[|x|]_{\xi;\rho} \in [|A|]_\xi} $.

We know that $x:A \in \Gamma$ from the \rulename{Var} rule.
So, $[|x|]_{\xi;\rho} =\rho(x)\in[|\Gamma(x)|]_\xi = [|A|]_\xi$.

\paragraph{Case (\rulename{Abs})}
We need to show that
$ \inference{\Delta;\Gamma |- \l x.t : A -> B & \xi;\rho \in [|\Delta;\Gamma|]}
	{[|\l x.t|]_{\xi;\rho} \in [|A -> B|]_\xi} $.

By induction, we know that
$ \inference{\Delta;\Gamma,x:A |- t : B & \xi';\rho' \in [|\Delta;\Gamma,x:A|]}
	     {[|t|]_{\xi';\rho'} \in [|B|]_\xi} $.

Since this holds for all $\xi';\rho' \in [|\Delta;\Gamma,x:A|]$, it also holds
for particular $\xi';\rho'$ such that $\xi'=\xi$ and
$\rho' = \rho[x \mapsto s]$ for any $s \in [|A|]_\xi' = [|A|]_\xi$.
Since saturated sets are closed under normalizing weak head expansion,
$(\l x.[|t|]_{\xi;\rho})~s \in[|B|]_\xi$ for any $s\in[|A|]_\xi$.
Therefore, $\l x.[|t|]_{\xi;\rho}$ is obviously in the very set,
which we wanted it to be in. That is,
\[
[|\l x.t|]_{\xi;\rho} = \l x.[|t|]_{\xi;\rho}
\in \{t\in \SN \mid t~s\in[|B|] ~\text{for all}~ s\in[|A|]\} 
= [|A -> B|]_\xi
\]

\paragraph{Case (\rulename{App})}
We need to show that
$ \inference{\Delta;\Gamma |- t~s : B & \xi;\rho\in[|\Delta;\Gamma|]}{[|t~s|]_{\xi;\rho} \in [|B|]_\xi} $.

By induction we know that
\[
\inference{\Delta;\Gamma |- t : A -> B & \xi;\rho\in[|\Delta;\Gamma|]}{[|t|]_{\xi;\rho} \in [|A -> B|]_\xi}
\qquad
\inference{\Delta;\Gamma |- s : A & \xi;\rho\in[|\Delta;\Gamma|]}{[|s|]_{\xi;\rho} \in [|A|]_\xi}
\]
Then, it is straightforward to see that $[|t~s|]_{\xi;\rho}\in[|B|]_\xi$
by definition of $[|A -> B|]_\xi$.

\paragraph{Case (\rulename{TyAbs})}
We need to show that
$ \inference{\Delta;\Gamma |- t : \forall X.B & \xi;\rho\in[|\Delta;\Gamma|]}
	{[|t|]_{\xi;\rho} \in [|\forall X.B|]_\xi} $

By induction, we know that
\[ \inference{\Delta,X;\Gamma |- t : B & \xi';\rho'\in[|\Delta,X;\Gamma|]}
	{[|t|]_{\xi';\rho'} \in [|B|]_{\xi'}} ~
	(X\notin\FV(\Gamma))
\]
Since this holds for all $\xi';\rho' \in [|\Delta,X;\Gamma|]$ where
$X\notin\FV(\Gamma)$, it also holds for particular subset such that
$\xi' = \xi[X\mapsto\mathcal{A}]$ and $\rho'=\rho$ for all $\mathcal{A}\in\SAT$.
That is,
\[ [|t|]_{\xi[X\mapsto\mathcal{A}];\rho} \in [|B|]_{\xi[X\mapsto\mathcal{A}]}
   \quad \text{for all}~\mathcal{A}\in\SAT \]
From $X\notin\FV(\Gamma)$, we know that
$[|t|]_{\xi[X\mapsto\mathcal{A}];\rho} = [|t|]_{\xi;\rho}$
because $\rho$ is independent of what $X$ maps to.
So, we know that
\[ [|t|]_{\xi;\rho} \in [|B|]_{\xi[X\mapsto\mathcal{A}]}
	\quad \text{for all}~\mathcal{A}\in\SAT \]
By set theoretic definition, this is exactly what we wanted to show:
\[ [|t|]_{\xi;\rho} \in
	\bigcap_{\mathcal{A}\in\SAT} [|B|]_{\xi[X\mapsto\mathcal{A}]}
	= [|\forall X.B|]_\xi
\]

\paragraph{Case (\rulename{TyApp})}
We need to show that
$ \inference{\Delta;\Gamma |- t : B[A/X] & \xi;\rho\in[|\Delta;\Gamma|]}
	{[|t|]_{\xi;\rho} \in [|B[A/X]|]_\xi} $

By induction, we know that
$ \inference{\Delta;\Gamma |- t : \forall X.B & \xi';\rho'\in[|\Delta;\Gamma|]}
	{[|t|]_{\xi';\rho'} \in [|\forall X.B|]_{\xi'}}
$.

Since this holds for all $\xi';\rho' \in [|\Delta,\Gamma|]$,
it also holds for $\xi'=\xi$ and $\rho'=\rho$. Then, we are done:
$ [|t|]_{\xi;\rho} \in [|\forall X.B|]_{\xi}
	= \bigcap_{\mathcal{A}\in\SAT} [|B|]_{\xi[X\mapsto\mathcal{A}]}
	\subseteq [|B|]_{\xi[X\mapsto[|A|]_\xi]} = [|B[A/X]|]_\xi
$.
\end{proof}
\begin{corollary}[strong normalization]
	$\inference{\Delta;\Gamma |- t : A}{t \in \SN}$
\end{corollary}
Once we have proved the soundness of typing with respect to interpretation,
it is easy to see that STLC is strongly normalizing by giving a trivial term
interpretation $\rho(x) = x$ for all the free variables.
Note that $[|t|]_{\xi;\rho} = t \in [|A|]_\xi \subset \SN$
under the trivial interpretation.

